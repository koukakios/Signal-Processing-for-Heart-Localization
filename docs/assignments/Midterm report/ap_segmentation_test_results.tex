\chapter{Test results classification}

\begin{table}[ht!]
\scriptsize
\centering
\captionsetup{justification=centering}
\begin{tabular}{|l|c|c|c|c|c|c|c|c|c|c|}
\hline
\textbf{Recording} & \textbf{Ch.} & \textbf{S1} & \textbf{S2} & \textbf{Uncertain} &
\multicolumn{2}{c|}{\textbf{False positive}} &
\multicolumn{2}{c|}{\textbf{False negative}} &
\multicolumn{2}{c|}{\textbf{Marked wrong}} \\
\cline{6-11}
 &  &  &  &  & S1 & S2 & S1 & S2 & S1 (is S2) & S2 (is S1) \\
\hline
Piezo 2 & 1 & 120 & 104 & 20 & 1 & 0 & 8 & 8 & 0 & 0 \\
Piezo 2 & 2 & 109 & 90 & 21 & 0 & 1 & 10 & 15 & 0 & 0 \\
Piezo 2 & 3 & 104 & 103 & 2 & 0 & 0 & 0 & 0 & 0 & 0 \\
Piezo 2 & 4 & 95 & 93 & 12 & 0 & 0 & 4 & 8 & 0 & 0 \\
Piezo 2 & 5 & 100 & 100 & 0 & 0 & 0 & 0 & 0 & 0 & 0 \\
\rowcolor{red!30}Piezo 2 & 6 & -- & -- & -- & 0 & 0 & 0 & 0 & 0 & 0 \\
\noalign{\hrule height 0.3pt}
Piezo 4 & 1 & 108 & 104 & 6 & 0 & 0 & 1 & 2 & 0 & 0 \\
Piezo 4 & 2 & 106 & 98 & 19 & 0 & 0 & 4 & 4 & 0 & 0 \\
Piezo 4 & 3 & 102 & 102 & 0 & 0 & 0 & 0 & 0 & 0 & 0 \\
\rowcolor{red!30}Piezo 4 & 4 & -- & -- & -- & 0 & 0 & 0 & 0 & 0 & 0 \\
Piezo 4 & 5 & 102 & 102 & 0 & 0 & 0 & 0 & 0 & 0 & 0 \\
\rowcolor{red!30}Piezo 4 & 6 & 24 & 19 & 83 & 0 & 0 & 78 & 78 & 0 & 0 \\
\hline
\textbf{Total} & & \textbf{946} & \textbf{896} & \textbf{80} &
\textbf{1} & \textbf{1} &
\textbf{27} & \textbf{37} &
\textbf{0} & \textbf{0} \\
\hline
\end{tabular}
\caption{Peak detection evaluation results for the piezo element. Rows colored red contain data the algorithm errored on. On those audio segments, a human eye could only -with great difficulty- detect the peaks and are therefore taken out of the total calculation.}
\label{tab:test_piezo_segmentation}
\end{table}

\begin{table}[ht!]
\scriptsize
\centering
\captionsetup{justification=centering}
\begin{tabular}{|l|c|c|c|c|c|c|c|c|c|c|}
\hline
\textbf{Recording} & \textbf{Ch.} & \textbf{S1} & \textbf{S2} & \textbf{Uncertain} &
\multicolumn{2}{c|}{\textbf{False positive}} &
\multicolumn{2}{c|}{\textbf{False negative}} &
\multicolumn{2}{c|}{\textbf{Marked wrong}} \\
\cline{6-11}
 &  &  &  &  & S1 & S2 & S1 & S2 & S1 (is S2) & S2 (is S1) \\
\hline
Stethoscope 2 & 1 & 111 & 110 & 0 & 0 & 0 & 0 & 0 & 0 & 0 \\
Stethoscope 2 & 2 & 110 & 109 & 2 & 0 & 0 & 1 & 2 & 0 & 0 \\
Stethoscope 2 & 3 & 111 & 110 & 0 & 0 & 0 & 0 & 0 & 0 & 0 \\
Stethoscope 2 & 4 & 111 & 110 & 0 & 0 & 0 & 0 & 0 & 0 & 0 \\
Stethoscope 2 & 5 & 111 & 110 & 0 & 0 & 0 & 0 & 0 & 0 & 0 \\
Stethoscope 2 & 6 & 111 & 110 & 2 & 0 & 0 & 1 & 1 & 0 & 0 \\
\noalign{\hrule height 0.3pt}
Stethoscope 5 & 1 & 113 & 112 & 0 & 0 & 0 & 0 & 0 & 0 & 0 \\
Stethoscope 5 & 2 & 114 & 110 & 9 & 0 & 0 & 4 & 3 & 0 & 0 \\
Stethoscope 5 & 3 & 113 & 112 & 0 & 0 & 0 & 0 & 0 & 0 & 0 \\
Stethoscope 5 & 4 & 113 & 112 & 0 & 0 & 0 & 0 & 0 & 0 & 0 \\
Stethoscope 5 & 5 & 115 & 113 & 2 & 0 & 0 & 0 & 0 & 0 & 0 \\
Stethoscope 5 & 6 & 113 & 112 & 0 & 0 & 0 & 0 & 0 & 0 & 0 \\
\hline
\textbf{Total} & & \textbf{1346} & \textbf{1330} & \textbf{15} &
\textbf{0} & \textbf{0} &
\textbf{6} & \textbf{6} &
\textbf{0} & \textbf{0} \\
\hline
\end{tabular}
\caption{Peak detection evaluation results for the stethoscope element. }
\label{tab:test_stethoscope_segmentation}
\end{table}
