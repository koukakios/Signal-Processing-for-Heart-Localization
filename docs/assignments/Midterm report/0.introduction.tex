\chapter{Introduction}

%Being a vital organ, the heart is of paramount importance in keeping a human being alive. Therefore, it is necessary to identify anomalies as fast and accurate as possible, to ensure quick and adequate treatment. A straightforward method to do this is using the sound waves that the heart produces. A doctor can do this manually using a stethoscope, but to achieve faster and objective data, a microphone array is superior. This project aims to use a microphone array consisting of 6 microphones to locate the heart valves of a patient and possibly give a diagnosis. 
% instructions on p. 83 of the manual state to not include the relevance of heart monitoring

This report describes a system which uses audio signals recorded by a microphone array placed on a human chest to analyze the heart. The aim is to localize the heart valves. The system is split into 4 modules: pre-processing, modeling, basic direction finding and advanced direction finding. From now the modules will be called module 1, module 2, module 3 and module 4, respectively. The goals of each module are as follows:
\begin{itemize}
    \item Module 1: This module takes the PCG data and pre-processes it, so that module 3 and 4 can use the data more easily.
    \item Module 2: This module creates a model for the heart. Using this, it is possible to create consistent data based on known parameters, which is useful in testing.
    \item Module 3: This module ...
    \item Module 4: This module ...
\end{itemize}

In Fig. \ref{fig:projectblock}, a block diagram of the system is given. Acquisition is performed by the microphone array, which captures sounds from the heart and saves the data. Pre-processing, corresponding with module 1, pre-processes that data and passes it on to data inspection, and localization, which is performed by modules 3 and 4. The possibilities exist to create a graphical user interface (GUI) for the latter 2 parts, and to create an algorithm that separates the sources. These two options will be carried out in case there is time left. 
This leads to an important point: the project has to be carried out in 10 weeks, with 4 team members, so it is important to stay realistic on what can be achieved and what not.



\begin{figure} [H]
    \centering
    \includegraphics[width=0.9\textwidth]{figures/projectblock.png}
    \caption{Overview of the system, as given in \cite{IP3Man}}  
    \label{fig:projectblock}
\end{figure}