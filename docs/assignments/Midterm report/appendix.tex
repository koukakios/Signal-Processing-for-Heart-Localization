
\chapter{ Block diagrams}
\begin{figure} [H]
    \centering
    \includegraphics[width=0.8\textwidth]{docs/assignments/Midterm report/figures/module_1/classification_flowchart.png}
    \caption{Block diagram of the steps taken by the functions "classify\_peaks" and "analyze\_diff2".}  
    \label{fig:classificationchart}
\end{figure}

\begin{figure} [H]
    \centering
    \includegraphics[width=0.8\textwidth]{docs/assignments/Midterm report/figures/module_1/solve_uncertains.png}
    \caption{Block diagram of the steps taken to increase the global threshold and the steps used by the function "solve\_uncertains".}  
    \label{fig:solve_uncertains_chart}
\end{figure}

\begin{figure} [H]
    \centering
    \includegraphics[width=0.8\textwidth]{docs/assignments/Midterm report/figures/module_2/multichannel_damping.png}
    \caption{Modeled signal for each microphone.}  
    \label{fig:multichannel_damping}
\end{figure}

\begin{figure} [H]
    \centering
    \includegraphics[width=0.8\textwidth]{docs/assignments/Midterm report/figures/module_2/multichannel_delay.png}
    \caption{Modeled signal for each microphone. Same data as Fig. \ref{fig:multichannel_damping}, but zoomed in.}  
    \label{fig:multichannel_delay}
\end{figure}


\begin{figure} [H]
    \centering
    \includegraphics[width=0.3\textwidth]{docs/assignments/Midterm report/figures/module_3/Beamformer/beamforming_pipeline.png}
    \caption{Description of the beamformer pipeline.}  
    \label{fig:beamformer_pipeline}
\end{figure}

\chapter{Module 3 Calculations}


Calculation for central frequency.
Given the spacing $d = 0.1$ m and assuming the speed of sound $v = 343$ m/s:
\begin{equation}
    f = \frac{v}{2d} = \frac{343}{2 \cdot 0.1} = \frac{343}{0.2} = 1715 \text{ Hz}
\end{equation}
\label{module_3_calculation_central_frequency}

\chapter{Module 3 figures}

\begin{figure} [H]
    \centering
    \includegraphics[width=0.8\textwidth]{docs/assignments/Midterm report/figures/module_3/Beamformer/Beamformer 1 source/Beamformer_30.png}
    \caption{Beamformer at $30^\circ$.}  
    \label{fig:beamformer_30_degrees}
\end{figure}

\begin{figure} [H]
    \centering
    \includegraphics[width=0.8\textwidth]{docs/assignments/Midterm report/figures/module_3/Beamformer/Beamformer 1 source/Beamformer_60.png}
    \caption{Beamformer at $30^\circ$.}  
    \label{fig:beamformer_60_degrees}
\end{figure}


\begin{figure} [H]
    \centering
    \includegraphics[width=0.8\textwidth]{docs/assignments/Midterm report/figures/module_3/MVDR/MVDR 1 source/MVDR_30.png}
    \caption{MVDR at $30^\circ$.}  
    \label{fig:MVDR_30_degrees}
\end{figure}

\begin{figure} [H]
    \centering
    \includegraphics[width=0.8\textwidth]{docs/assignments/Midterm report/figures/module_3/MVDR/MVDR 1 source/MVDR_60.png}
    \caption{MVDR at $60^\circ$.}  
    \label{fig:MVDR_60_degrees}
\end{figure}

\begin{figure} [H]
    \centering
    \includegraphics[width=0.8\textwidth]{docs/assignments/Midterm report/figures/module_3/MVDR/MVDR 2 source/MVDR_7_-7.png}
    \caption{MVDR at $60^\circ$.}  
    \label{fig:MVDR_2_sources}
\end{figure}

\chapter{Module 4 figures}
\begin{figure} [H]
    \centering
    \includegraphics[width=0.8\textwidth]{docs/assignments/Midterm report/figures/module_4/SVD_explanation.png}
    \caption{SVD interpretation.}  
    \label{fig:SVD_interpretation}
\end{figure}

\begin{figure} [H]
    \centering
    \includegraphics[width=0.8\textwidth]{docs/assignments/Midterm report/figures/module_4/MUSIC/MUSIC Single Source/Music_30_degrees.png}
    \caption{Beamformer at +7$^\circ$, -7$^\circ$, higher resolution.}  
    \label{fig:music_1_source_30_degrees}
\end{figure}

\begin{figure} [H]
    \centering
    \includegraphics[width=0.8\textwidth]{docs/assignments/Midterm report/figures/module_4/MUSIC/MUSIC Single Source/Music_60_degrees.png}
    \caption{Beamformer at +7$^\circ$, -7$^\circ$, higher resolution.}  
    \label{fig:music_1_source_60_degrees}
\end{figure}


\begin{figure} [H]
    \centering
    \includegraphics[width=0.8\textwidth]{docs/assignments/Midterm report/figures/module_4/MUSIC/MUSIC 2 sources/Bin freq = 1687 no dashed line.png}
    \caption{Beamformer at +7$^\circ$, -7$^\circ$, higher resolution.}  
    \label{fig:music_2_source_1687}
\end{figure}


\section{Assignments module 3 and 4}
\subsection{Assignment 6.2.3}

\subsection{Assignment 6.2.6}
\label{assigment_6.2.6}
For direction of arrival at $0^\circ$, $M = 7$, $f = 500\text{Hz}$, 
and $v = 340\text{m/s}$, the following plots have been acquired. 
It is obvious that the resolution increases (sharper lobes) but 
aliasing is introduced with increasing $\Delta$.

\begin{figure} [H]
    \centering
    \includegraphics[width=0.8\textwidth]{docs/assignments/Midterm report/figures/Assignments/assignment_6.2.6.1.png}
    \caption{}  
    \label{fig:assignment_6.2.6.1}
\end{figure}

\begin{figure} [H]
    \centering
    \includegraphics[width=0.8\textwidth]{docs/assignments/Midterm report/figures/Assignments/assignment_6.2.6.2.png}
    \caption{}  
    \label{fig:assignment_6.2.6.2}
\end{figure}

\begin{figure} [H]
    \centering
    \includegraphics[width=0.8\textwidth]{docs/assignments/Midterm report/figures/Assignments/assignment_6.2.6.3.png}
    \caption{}  
    \label{fig:assignment_6.2.6.3}
\end{figure}


For a different angle , e.g. $30^\circ$, the following plots were obtained, confirming that aliasing might lead to misleading information.

\begin{figure} [H]
    \centering
    \includegraphics[width=0.8\textwidth]{docs/assignments/Midterm report/figures/Assignments/Assigment 6.2.6/30 degrees.png}
    \caption{}  
    \label{fig:assignment_6.2.6.30a}
\end{figure}

\begin{figure} [H]
    \centering
    \includegraphics[width=0.8\textwidth]{docs/assignments/Midterm report/figures/Assignments/Assigment 6.2.6/30 degrees 2.png}
    \caption{}  
    \label{fig:assignment_6.2.6.30b}
\end{figure}

\begin{figure} [H]
    \centering
    \includegraphics[width=0.8\textwidth]{docs/assignments/Midterm report/figures/Assignments/Assigment 6.2.6/30 degrees 3.png}
    \caption{}  
    \label{fig:assignment_6.2.6.30c}
\end{figure}


\subsection{Assignment 6.2.9}

\begin{enumerate}
    \item \textbf{Narrowband Condition (Linear Array)} \\
    The text states the condition is $B \ll v/D$, where $D = (M-1)d$. \\
    Given: $M=6$, $d=0.1$ m, $v \approx 340$ m/s (speed of sound).
    $$ D = (6-1)(0.1) = 0.5 \text{ m} $$
    $$ B \ll \frac{340}{0.5} = 680 \text{ Hz} $$
    
    \item \textbf{Minimal Spacing for Spatial Aliasing} \\
    The text states we need $\Delta < 1/2$, which means $d < \lambda/2$. \\
    Given: $F_0 = 1000$ Hz.
    $$ \lambda = \frac{v}{F_0} = \frac{340}{1000} = 0.34 \text{ m} $$
    $$ d < \frac{0.34}{2} = 0.17 \text{ m} $$
    The spacing must be less than 17 cm.

    \item \textbf{Heart Sound Array ($2 \times 3$)} \\
    We need $B \ll v_{body}/D_{max}$. \\
    Given: Spacing $d=0.05$ m, $v_{body} = 60$ m/s. \\
    The array is 2 rows by 3 columns. The max distance $D$ is the diagonal from corner to corner.
    $$ \text{Width} = (3-1)d = 2d, \quad \text{Height} = (2-1)d = 1d $$
    $$ D = \sqrt{(2d)^2 + (d)^2} = d\sqrt{5} \approx 0.05 \times 2.236 = 0.112 \text{ m} $$
    $$ B \ll \frac{60}{0.112} \approx 536 \text{ Hz} $$

    \item \textbf{Physical Reason for $\Delta$} \\
    $\Delta = d/\lambda$ represents the sampling density in space.
    \begin{itemize}
        \item \textbf{Smaller $\Delta$:} Samples are very close. This reduces the phase difference between microphones, making the main lobe wider (worse resolution).
        \item \textbf{Larger $\Delta$:} Samples are far apart. This improves resolution (sharper lobe), but if $\Delta > 0.5$, the sensors are too far apart to distinguish the wave cycle uniqueley (undersampling), causing spatial aliasing (false peaks).
    \end{itemize}
\end{enumerate}